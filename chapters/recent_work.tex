\section{Recent Work \& Future of Ransomware}

With the growing impact of ransomware, the research about it also grew. Many recent papers suggest methods to automatically detect new ransomware. A common approach here is to create some kind of virtual environment to measure file I/O activities for the detection. \cite{Ahmadian2016} hereby focuses on detecting high survivable ransomware, where 20 indicators like access to cryptographic libraries and access to specific registry path keys are used. They are divided into static and dynamic analysis types. \cite{Shukla2016} proposes a similiar approach, which in addition uses a behavioural model to learn new ransomware behaviour while under attack. Furthermore the system of \cite{Kirda2017} measures I/O data buffer entropy and constructs file access pattern to detect file lockers and takes automatic screenshots to detect screen lockers. It relies on the fact that ransomware consists of specific behaviours which are difficult to change like malicious encryption and desktop locking. The PayBreak paper \cite{Kolodenker2017} on the other hand provides a method to store session keys for every encryption inside a vault, so that files can be decrypted without paying a ransom. In \cite{Moore2016} a Honeypot technique is used to detect ransomware activity, but the paper concludes that there is no guarantee that the ransomware will attempt to invade decoy areas. \cite{Continella2016} even proposes a ransomware-aware filesystem, where the operating system is responsible to keep an always-fresh, automatic backup of the files which were recently modified. Thus they also try to avoid the traditional backup performance trade off.\\
\\
Another research field around ransomware is devoted to mobile devices, where mainly the Android operating system is examined. \cite{Yang2015} and \cite{Andronio2015} both propose a system, where the compressed application file (\texttt{.apk}) is analyzed and reverse-engineered to detect malicious behaviour. In \cite{Maiorca2017} this is enhanced by counting each system API package used by the application and putting the result together with the rest of the analyzed data into a mathematical function whose parameters have already been tuned in a training phase beforehand.\\
\\
Many other papers address the evolution, mitigation and prevention of ransomware. \cite{Shinde2016} and \cite{Richardson2017} e.g. illustrate that existing mitigation strategies work well enough, but too few people and companies are making (correct) use of them.\\
\\
Right now there is no well-established concept to protect users, both business and private ones, from novel ransomware attacks. Of course general prevention and mitigation methods like automated backups, automated system patches and education can help to decrease the risk, but these techniques are not well implemented by everybody, which was shown by the attacks of the recent years. It will be interesting to see, whether some kind of automatic detection of new ransomware will be part of an anti-virus software. \\
\\
Some recent technologies can also influence ransomware in the future. Namely these are:

\begin{description}
\item[\gls{iot}:] Right now ransomware mainly infects servers, personal computers and mobile phones. Attackers could also target more things connected to the internet like ``smart watches'' or ``connected cars'' \cite{Cobb2017}. Security updates for specific firm wares are often not as soon released as widely-spread operating system updates, if at all.
\item[Smart Contracts:] Smart contracts are protocols, which are intended to facilitate, verify or enforce real world contracts. On some blockchains like Ethereum smart contracts can be defined to interact with crypto currencies. Together with a custodian that can be implemented using tamper-resistant hardware or cryptographic obfuscation techniques, so called ``autonomous ransomware'' could be possible \cite{Kaptchuk2017}\cite{Juels2016}. With this there is no need for a \gls{candc} server, because the infection as well as the decryption can happen autonomously.
\end{description} 

